% Options for packages loaded elsewhere
\PassOptionsToPackage{unicode}{hyperref}
\PassOptionsToPackage{hyphens}{url}
%
\documentclass[
]{article}
\usepackage{lmodern}
\usepackage{amssymb,amsmath}
\usepackage{ifxetex,ifluatex}
\ifnum 0\ifxetex 1\fi\ifluatex 1\fi=0 % if pdftex
  \usepackage[T1]{fontenc}
  \usepackage[utf8]{inputenc}
  \usepackage{textcomp} % provide euro and other symbols
\else % if luatex or xetex
  \usepackage{unicode-math}
  \defaultfontfeatures{Scale=MatchLowercase}
  \defaultfontfeatures[\rmfamily]{Ligatures=TeX,Scale=1}
\fi
% Use upquote if available, for straight quotes in verbatim environments
\IfFileExists{upquote.sty}{\usepackage{upquote}}{}
\IfFileExists{microtype.sty}{% use microtype if available
  \usepackage[]{microtype}
  \UseMicrotypeSet[protrusion]{basicmath} % disable protrusion for tt fonts
}{}
\makeatletter
\@ifundefined{KOMAClassName}{% if non-KOMA class
  \IfFileExists{parskip.sty}{%
    \usepackage{parskip}
  }{% else
    \setlength{\parindent}{0pt}
    \setlength{\parskip}{6pt plus 2pt minus 1pt}}
}{% if KOMA class
  \KOMAoptions{parskip=half}}
\makeatother
\usepackage{xcolor}
\IfFileExists{xurl.sty}{\usepackage{xurl}}{} % add URL line breaks if available
\IfFileExists{bookmark.sty}{\usepackage{bookmark}}{\usepackage{hyperref}}
\hypersetup{
  pdftitle={Czy możemy już obwiniać nasz rząd?},
  pdfauthor={Agnieszka Choczyńska},
  hidelinks,
  pdfcreator={LaTeX via pandoc}}
\urlstyle{same} % disable monospaced font for URLs
\usepackage[margin=1in]{geometry}
\usepackage{longtable,booktabs}
% Correct order of tables after \paragraph or \subparagraph
\usepackage{etoolbox}
\makeatletter
\patchcmd\longtable{\par}{\if@noskipsec\mbox{}\fi\par}{}{}
\makeatother
% Allow footnotes in longtable head/foot
\IfFileExists{footnotehyper.sty}{\usepackage{footnotehyper}}{\usepackage{footnote}}
\makesavenoteenv{longtable}
\usepackage{graphicx,grffile}
\makeatletter
\def\maxwidth{\ifdim\Gin@nat@width>\linewidth\linewidth\else\Gin@nat@width\fi}
\def\maxheight{\ifdim\Gin@nat@height>\textheight\textheight\else\Gin@nat@height\fi}
\makeatother
% Scale images if necessary, so that they will not overflow the page
% margins by default, and it is still possible to overwrite the defaults
% using explicit options in \includegraphics[width, height, ...]{}
\setkeys{Gin}{width=\maxwidth,height=\maxheight,keepaspectratio}
% Set default figure placement to htbp
\makeatletter
\def\fps@figure{htbp}
\makeatother
\setlength{\emergencystretch}{3em} % prevent overfull lines
\providecommand{\tightlist}{%
  \setlength{\itemsep}{0pt}\setlength{\parskip}{0pt}}
\setcounter{secnumdepth}{-\maxdimen} % remove section numbering

\title{Czy możemy już obwiniać nasz rząd?}
\author{Agnieszka Choczyńska}
\date{}

\begin{document}
\maketitle

\hypertarget{rzux105dy-w-czasach-zarazy}{%
\subsection{Rządy w czasach zarazy}\label{rzux105dy-w-czasach-zarazy}}

Epidemia nowego koronawirusa uruchomiła w nas rzadko spotykaną zgodność.
Ogólnie się uważa, że trzeba myć ręce, zasłaniać twarze i izolować się
od ludzi, ponieważ mamy powszechne przekonanie, że to, jak długo potrwa
i jak dotkliwy będzie ten kryzys, zależy w największej mierze od nas
samych. Można spotkać się z opiniami - jak ta na zdjęciu - że
rozprzestrzenianie wirusa zależy tylko od dwóch czynników:

\begin{figure}
\centering
\includegraphics{joke.jpg}
\caption{\emph{Gra słów: przymiotnik ``dense'' może oznaczać zarówno
``gęsty'' jak i ``tępy''.}}
\end{figure}

\href{https://www.boredpanda.com/coronavirus-covid-19-jokes/?utm_source=pl.pinterest\&utm_medium=referral\&utm_campaign=organic}{źródło}

Ale przede wszystkim, kryzys daje to okazję do ciskania gromów na
rządzących. Jedni kochają nienawidzić swoich decydentów, inni kochają
nienawidzić wszystkich pozostałych - a epidemia dostarcza wszystkim
świeżych materiałów. W tej analizie przyjrzę się sytuacji państw UE,
porównując, jak reagowały na zagrożenie i jak sobie z nim radziły.

\hypertarget{problemy}{%
\subsection{Problemy}\label{problemy}}

Odpowiedzi na te pytania nie są wcale proste do znalezienia. Po drodze
trzeba zmierzyć się z szeregiem problemów i podjąć sporo dość
uznaniowych decyzji. W ostateczności ocena zawsze zależy od wielu
kryteriów i każdy może ją wystawić, kierując się własnymi priorytetami.

\hypertarget{dane}{%
\subsubsection{Dane}\label{dane}}

Epidemia nie jest eksperymentem ani zaplanowaną inwestycją, więc nie
mógł powstać na jej potrzeby przemyślany i zunifikowany system zbierania
danych. Istniejące już instytucje robią świetną robotę, ale trzeba się
liczyć z brakami, możliwością nieścisłości, różnicami w definiowaniu
wskaźników albo niepełnym zasięgiem geograficznym.

\hypertarget{co-oznacza-sukses-albo-poraux17cka}{%
\subsubsection{Co oznacza sukses albo
porażka?}\label{co-oznacza-sukses-albo-poraux17cka}}

Każdy możliwy sposób mierzenia suksesu w walce z epidemią ma wady i
ograniczenia. Istnieją spisy potwierdzonych przypadków zarażenia
wirusem, ale te liczby nie odzwierciedlają przecież rzeczywistej liczby
przypadków - można by je zredukować do zera, po prostu nie
przeprowadzając testów. Liczba śmierci może też zależeć od tego, w
jakich przypadkach uznajemy, że to właśnie wirus był przyczyną zgonu.
Zawsze gdy go stwierdzono, czy tylko gdy nie było poważniejszej choroby
towarzyszącej?

Najbardziej intuicyjne wydawałoby się porównanie przeciętnej
śmiertelności w latach poprzednich i oszacowanie ``nadliczbowych
śmierci''. Wysoka liczba oznaczałaby bezsilność wobec wirusa, bliska
zeru: że środki zaradcze okazały się skuteczne, a ujemna: że być może
były po części zbędne. Problem w tym, że dane demograficzne na ten okres
nie są powszechnie dostępne. To samo dotyczy wskaźników
makroekonomicznych, jak PKB czy bezrobocie.

\hypertarget{kto-za-to-odpowiada}{%
\subsubsection{Kto za to odpowiada?}\label{kto-za-to-odpowiada}}

Inną kwestią jest, na ile działania państwa mają wpływ na sukses lub
porażkę w tej walce. Niektóre tragedie są nie do uniknięcia, a ich
łagodniejszy przebieg może wynikać z uprzywilejowanej z jakiegoś powodu
pozycji państwa, a nie jego sprawności i zapobiegliwości. Nie wszystkie
działania rządów jesteśmy też w stanie przedstawić liczbowo, część
szczegółów zgubi się w koniecznej generalizacji, a nawet dobre pomysły
mogą spalić na panewce, jeśli obywatele się do nich nie zastosują.

\hypertarget{rzut-oka-na-sytuacjux119}{%
\subsection{Rzut oka na sytuację}\label{rzut-oka-na-sytuacjux119}}

Ponieważ chcę porównywać sytuację w różnych państwach, wszystkie liczby
- zarażonych, zmarłych czy wykonanych testów - będą podane relatywnie,
na 1 mln mieszkańców. To samo będzie dotyczyło sytuacji gospodarczej,
gdzie sumy wydane na zapobiegnięcie kryzysowi będą podane w relacji do
PKB.

W tabeli poniżej przedstawiono daty pojawienia się (a właściwie
stwierdzenia obecności) wirusa w kolejnych państwach. Dane były zbierane
przez organizację Our World In Data
(\href{https://github.com/owid/covid-19-data}{źródło}).

\begin{longtable}[]{@{}rll@{}}
\caption{Moment pojawienia się pierwszego przypadku}\tabularnewline
\toprule
nr & Państwo & Data\tabularnewline
\midrule
\endfirsthead
\toprule
nr & Państwo & Data\tabularnewline
\midrule
\endhead
1 & France & 2020-01-25\tabularnewline
2 & Germany & 2020-01-28\tabularnewline
3 & Finland & 2020-01-30\tabularnewline
4 & Italy & 2020-01-31\tabularnewline
5 & Spain & 2020-02-01\tabularnewline
6 & Belgium & 2020-02-04\tabularnewline
7 & Austria & 2020-02-26\tabularnewline
8 & Switzerland & 2020-02-26\tabularnewline
9 & Denmark & 2020-02-27\tabularnewline
10 & Greece & 2020-02-27\tabularnewline
11 & Romania & 2020-02-27\tabularnewline
12 & Estonia & 2020-02-28\tabularnewline
13 & Netherlands & 2020-02-28\tabularnewline
14 & Ireland & 2020-03-01\tabularnewline
15 & Luxembourg & 2020-03-01\tabularnewline
16 & Portugal & 2020-03-01\tabularnewline
17 & Czech Republic & 2020-03-02\tabularnewline
18 & Poland & 2020-03-03\tabularnewline
19 & Hungary & 2020-03-04\tabularnewline
20 & Slovenia & 2020-03-05\tabularnewline
21 & Bulgaria & 2020-03-08\tabularnewline
22 & Cyprus & 2020-03-10\tabularnewline
\bottomrule
\end{longtable}

Zgodnie z tymi danymi, wirus pojawił się w Europie w ostatnim tygodniu
stycznia. W ciągu dziesięciu dni opanował sześć państw, głównie z
połudiowo-zachodniej Europy. Druga fala zaczęła się pod koniec lutego,
niemal codziennie ogarniając kolejny kraj. W Polsce wirus pojawił się
bardzo późno, dopiero w marcu.

\hypertarget{potwierdzone-przypadki}{%
\subsubsection{Potwierdzone przypadki}\label{potwierdzone-przypadki}}

Na wykresach przedstawiono liczbę potwierdzonych przypadków na 1 mln
mieszkańców w każdym z państw w kolejnych dniach. Dla poprawienia
czytelności, państwa podzielono według tego, jak wcześnie pojawił się w
nich wirus. W pierwszej grupie znalazły się te, w których pierwszy
przypadek stwierdzono na przełomie stycznia i lutego, w drugiej - pod
koniec lutego, a trzecia obejmuje kraje, które epidemia dotknęła dopiero
w marcu.

\includegraphics{article_files/figure-latex/cases-1.pdf}
\includegraphics{article_files/figure-latex/cases-2.pdf}
\includegraphics{article_files/figure-latex/cases-3.pdf}

W najwcześniejszej grupie większość znajduje się między 1000 a 5000
przypadków na 1 mln (0,1 - 0,5\%). Przez większość marca prowadziły
Włochy, w ostatnim tygodniu wyprzedzone przez Hiszpanię.

W drugiej grupie pięć państw ma mniej niż 0,2\% przypadków. Mocno wybija
się Szwajcaria - być może dlatego, że jest państwem małym (obiektywny
nieduży przyrost zachorowań oznacza relatywnie wysoki wzrost) i
praktycznie otoczonym państwami z pierwszej grupy.

Zdecydowana większość państw ostatniej grupy ma wciąż poniżej 0,1\%
zarażonych. Co ciekawe, w tej grupie znajduje się państwo o najwyższym w
ogóle odsetku zarażonych - Luksemburg. Trzeba jednak wziąć pod uwagę, że
ma około 600,000 ludności (więc \emph{rzeczywistych} przypadków jest
mniej niż przypadków na 1 mln) i leży pomiędzy Niemcami a Francją.

\hypertarget{testowanie}{%
\subsubsection{Testowanie}\label{testowanie}}

{[}Dane o liczbie wykonanych testów częściowo uzupełniałam z
wykorzystaniem strony
\href{worldometers.info/coronavirus/?utm_campaign=homeAdvegas1?}{worldometer}.{]}

Liczba znanych przypadków zakażenia zależy jednak w dużej mierze od
tego, jak rzetelnie staramy się je znaleźć, czyli od liczby i sposobu
przeprowadzania testów. Pojawiają się tutaj trzy zasadnicze problemy:

\begin{itemize}
\tightlist
\item
  Braki danych - o ile liczby zakażeń i śmierci są podawane dość
  rzetelnie przez wszystkie państwa, to liczby wykonywanych testów już
  nie są. Brakujące dane uzupełniono liniową interpolacją, przyjmując -
  jeśli to było konieczne - że pierwsze testy wykonano w dniu pojawienia
  się pierwszych przypadków i w liczbie odpowiadającej tym przypadkom.
  To niemal zawsze zaniża liczbę testów. Na górnym wykresie podano
  odsetek znanych wartości liczby testów. Im jest większy, tym bardziej
  wiarygodne dane.
\item
  Polityka testowania - w niektórych państwach testy są wykonywane na
  szerokiej grupie, w innych tylko na ludziach, którzy mieli wysokie
  prawdopodobieństwo zetknięcia się z wirusem, albo nawet wyłącznie na
  tych, którzy wykazują objawy. Ze statystycznego punktu widzenia
  najlepsza jest ta pierwsza polityka testowania - im bardziej losowa
  grupa testowa, tym bardziej znana liczba zarażonych odzwierciedla stan
  faktyczny. Na dolnym wykresie przedstawiono przeciętny odsetek
  pozytywnych testów. Im jest niższy, tym bardziej ``szerokie''
  testowanie.
\end{itemize}

Innym kolorem oznaczono Francję, gdzie odsetek pozytywnych testów okazał
się większy od 1 (dla czytelności słupek na wykresie został obcięty).
Wynika to z tego, że w pierwszych dniach rozwoju epidemii liczba testów
jest na poziomie zera, mimo, że zarejestrowano już przypadki zarażenia.
Trzeba to potraktować jako błąd w danych.

\includegraphics{article_files/figure-latex/testing-1.pdf}

\begin{itemize}
\tightlist
\item
  Trzecia kwestia dotyczy interpretacji liczby potwierdzonych
  przypadków. Jeśli państwo X ma ich więcej niż państwo Y, to oznacza,
  że X znajduje się w gorszej epidemicznie sytuacji, czy raczej, że
  bardziej jest swojej prawdziwej sytuacji świadome? Jeśli dla każdego
  kraju wezmę liczbę przypadków i liczbę testów (w najwyższym punkcie) i
  policzę korelację, to wynosi ona 0.6524. Oznacza to, że \textbf{65\%
  zmienności maksymalnej liczby przypadków można wytłumaczyć zmiennością
  liczby testów, a nie różnicą w stopniu rozwoju epidemii}. Określenia
  takie jak ``dużo'' czy ``mało'' są zawsze uznaniowe, ale moim zdaniem
  to zdecydowanie za dużo, żeby uznać liczbę potwierdzonych przypadków
  za wskaźnik sytuacji państwa, zwłaszcza przy porównaniu z innymi.
\end{itemize}

Spróbuję teraz odfiltorwać ten efekt. Rozważmy model:

\[\ C = \alpha + \beta * T + \epsilon \] gdzie: C - liczba przypadków, T
- liczba testów.

\(\beta\) jest współczynnikiem, prawdopodobnie dużo mniejszym od 1,
ponieważ z reguły T \textgreater\textgreater{} C. Oznacza on, jak duża
część zmienności liczby przypadków wynika ze zmienności liczby testów.
Innymi słowy: o ile większa byłaby liczba znanych przypadków, gdyby
liczba testów była większa o jednostkę, \textbf{przy tym samym stanie
faktycznym epidemii}. Od każdego C można odjąć człon \(\beta * T\),
uzyskując ``czystą'' liczbę przypadków.

Tak zmodyfikowana zmienna traci swoją podstawową interpretację - może
być nawet ujemna. Ale przy porównaniach sytuacji nie faworyzuje państw,
które przeprowadzają mniej testów.

Tak wygląda wyestymowany model:

\begin{verbatim}
## 
## Call:
## lm(formula = cases ~ tests_mod, data = cases_modified)
## 
## Residuals:
##      Min       1Q   Median       3Q      Max 
## -2868.45   -65.03     2.66     2.66  2499.64 
## 
## Coefficients:
##              Estimate Std. Error t value Pr(>|t|)    
## (Intercept) -2.659533  12.769784  -0.208    0.835    
## tests_mod    0.086426   0.001117  77.384   <2e-16 ***
## ---
## Signif. codes:  0 '***' 0.001 '**' 0.01 '*' 0.05 '.' 0.1 ' ' 1
## 
## Residual standard error: 524.3 on 2258 degrees of freedom
## Multiple R-squared:  0.7262, Adjusted R-squared:  0.7261 
## F-statistic:  5988 on 1 and 2258 DF,  p-value: < 2.2e-16
\end{verbatim}

(Analogicznie rozważyłam również wpływ struktury demograficznej
społeczeństwa - odsetek osób po 60. roku życie - ale okazał się
nieistotny.)

\hypertarget{kolejka}{%
\subsubsection{Kolejka}\label{kolejka}}

Kolejną kwestią, którą trzeba uwzględnić, jest czas. Trudno szczycić się
lekkim przebiegiem epidemii z pozycji kraju, które leży na geograficznym
poboczu i miało dwa miesiące więcej na reakcję.

Wyobraźmy sobie, że ustawiamy kraje w kolejności według pojawienia się
pierwszego przypadku, oraz w drugiej kolejce, według ``oczyszczonej''
liczby przypadków. Jeśli w paru krajach epidemia zaczęła się w tym samym
dniu, wyższe miejsce będzie miał ten, w którym tego dnia pojawiło się
więcej przypadków.

Gdyby w każdym kraju epidemia rozwijała się w takim samym tempie, każde
państwo zajęłoby to samo miejsce w obu kolejkach. Tak jednak nie jest:
znany jest choćby przykład Hiszpanii, która zaczęła później niż Włochy,
a prześcignęła je pod względem liczby przypadków. Proponuję zatem
obliczyć \textbf{wskaźnik rozwoju epidemii} (względem innych państw)
jako różnicę między miejscem w kolejce czasu i kolejce przypadków.

Jeśli w jakimś państwie epidemia rozwija się szybciej niż w państwie
poprzednim (z wcześniejszą datą pierwszego przypadku), to ``wyprzedza''
je w kolejce i jego indeks wzrasta o 1. Państwa o ujemnym indeksie mają
z kolei stosunkowo mało przypadków, biorąc pod uwagę moment, w którym
zaczęły. Epidemia rozwija się tam wolniej, z czego wyciągamy wniosek, że
jest \emph{skuteczniej hamowana} (niekoniecznie prawdziwy, bo może to
wynikać z niezależnych od człowieka czynników albo być dziełem
przypadku, ale tak trzeba założyć w podstawowej analizie).

Wykres poniżej przedstawia wartości indeksu.

\includegraphics{article_files/figure-latex/rangs-1.pdf}

Zdecydowanie najlepiej wypadają Niemcy, które zaczęły epidemię jako
jedne z pierwszych, a pod względem przypadków znalazły się blisko końca.
Bardzo dobry wynik ma też Finlandia. Polska również znalazła się ``po
dobrej stronie mocy'', z wartością indeksu -4.

\hypertarget{ux15bmiertelnoux15bux107}{%
\subsubsection{Śmiertelność}\label{ux15bmiertelnoux15bux107}}

Inną kwestią jest śmiertelność choroby powodowanej przez wirusa, czyli
stosunek liczby śmierci do liczby zarażonych. Jeżeli założymy, że:

\begin{itemize}
\tightlist
\item
  w analizowanych państwach mamy do czynienia z tym samym wirusem, a
  nie, np. w niektórych z groźniejszą mutacją;
\item
  śmierci pacjentów z pozytywnym wynikiem testu są wszędzie traktowane
  jako śmierci związane z wirusem, niezależnie od innych chorób;
\item
  na śmiertelność nie wpływają istotnie inne czynniki, np. środowiskowe
  czy demograficzne;
\end{itemize}

to różnice w poziomie śmiertelności mogą odzwierciedlać wydajność służby
zdrowia w poszczególnych państwach. Im bardziej zaawansowana i dostępna
opieka medyczna, tym mniej przypadków choroby skończy się tragicznie.

Rzetelne zweryfikowanie tych założeń wymaga eksperckiej wiedzy, ale
jedną rzeczą, którą na pewno warto sprawdzić, jest struktura
demograficzna. Wiadomo, że najciężej chorobę przechodzą osoby starsze,
dlatego większa śmiertelność może wynikać wprost z większego odsetka
takich osób w społeczeństwie. Analogicznie jak dla liczby przypadków i
testowania, można przygotować model:

\begin{verbatim}
## 
## Call:
## lm(formula = death_rate ~ elders, data = mortality)
## 
## Residuals:
##      Min       1Q   Median       3Q      Max 
## -0.03773 -0.03025 -0.01049  0.01147  0.15485 
## 
## Coefficients:
##               Estimate Std. Error t value Pr(>|t|)    
## (Intercept) -0.0048916  0.0091214  -0.536    0.592    
## elders       0.0018693  0.0004642   4.027 5.93e-05 ***
## ---
## Signif. codes:  0 '***' 0.001 '**' 0.01 '*' 0.05 '.' 0.1 ' ' 1
## 
## Residual standard error: 0.03836 on 1486 degrees of freedom
## Multiple R-squared:  0.0108, Adjusted R-squared:  0.01013 
## F-statistic: 16.22 on 1 and 1486 DF,  p-value: 5.927e-05
\end{verbatim}

Odsetek starszych osób w społeczeństwach nie wyjaśnia dobrze różnic w
śmiertelności (\(R^2\) jest bardzo niskie), co oznacza, że nie można
zrzucić całej winy za wyższą śmiertelność na różnice demograficzne. Być
może wynika to z faktu, że badane kraje nie są zbyt mocno zróżnicowane
pod tym względem. Niemniej, demografia jest istotna, więc nie zaszkodzi
``oczyścić'' wskaźnika śmiertelności, podobnie, jak to było robione
wcześniej.

Drugą rzeczą, którą trzeba uwzględnić, jest stosunek śmiertelności do
liczby przypadków. Służby zdrowia - liczby łóżek, personelu i sprzętu -
nie da się łatwo skalować. Nawet dobrze zorganizowany system może nie
wytrzymać nagłego, dużego napływu pacjentów. I \emph{vice versa}, nawet
niezbyt sprawny poradzi sobie z nielicznymi przypadkami.

Wykres poniżej pokazuje tę zależność. Czerwonymi liniami oznaczono
mediany obu zmiennych, dodatkowo żółtymi kwantyle 25\% i 75\%.

\includegraphics{article_files/figure-latex/mortality-1.pdf}

Jeśli podzielimy przestrzeń wykresu według czerwonych linii na cztery
części, to w prawym dolnym rogu znajdą się państwa, które mają niską
śmiertelność mimo wysokiej liczby przypadków - Szwajcaria, Luksemburg,
Portugalia, Austria i Niemcy. Ich służba zdrowia jest na tyle sprawna,
że jest w stanie ratować pacjentów nawet przy dużej ich liczbie.

W lewym dolnym rogu mamy państwa, które uzyskały niską śmiertelność, ale
też nie musiały się zmierzyć z aż tak dużym wyzwaniem. Z kolei w prawym
górnym takie, w których śmiertelność jest wysoka, ale poniekąd
usprawiedliwona skalą epidemii.

I wreszcie w lewym górnym rogu można znaleźć państwa o wysokiej
śmiertelności mimo małej skali zachorowań - Węgry, Bułgaria, Grecja,
Polska, Rumunia, Słowenia i Cypr.

\hypertarget{gospodarka}{%
\subsubsection{Gospodarka}\label{gospodarka}}

Jedną z głównych osi napięcia jest ta między restrykcjami związanymi z
epidemią a potrzebami gospodarki. Zamykanie lokali usługowych,
wstrzymanie ruchu turystycznego, zwalnianie pracowników, którzy nie mogą
realizować swojej pracy w domach - to wszystko w oczywisty sposób
prowadzi do spowolnienia.

Ponownie natrafiamy tu na szereg problemów. Makroekonomiczne wskaźniki -
jak bezrobocie czy PKB - nie są obliczane na bieżąco. Oczywistym
rozwiązaniem są giełdy, publikujące notowania codziennie. Indeksy, takie
jak WIG, odzwierciedlają ogólny stan giełdy i są używane jako wskaźniki
stanu gospodarki, ale wciąż trzeba pamiętać, że gospodarka nie sprowadza
się do spółek akcyjnych. A także, że indeksy są konstruowane w różny
sposób i niekoniecznie w pełni porównywalne.

Drugi problem jest taki, że trudno znaleźć obiektywny punkt odniesienia.
Na wykresie porównano procentowe przyrosty (wszystkie są ujemne)
głównych indeksów giełdowych między końcem zeszłego roku (30.12.19) a
końcem badanego okresu (30.04.20) oraz najniższym notowaniem osiągniętym
w całym tym okresie (dla większości było to w marcu). Widać, że spadek
mógł sięgać niemal 50\%. Niektóre państwa, jak Dania, zdążyły do końca
kwietnia skoczyć do góry. Inne, jak Cypr, pozostały na podobnym
poziomie.

\includegraphics{article_files/figure-latex/stock-1.pdf}

\hypertarget{front-walki-z-wirusem}{%
\subsection{Front walki z wirusem}\label{front-walki-z-wirusem}}

Przechodzimy wreszcie do analizy działań rządów. Zbiór danych został
przygotowany w ramach projektu \emph{Oxford COVID-19 Government Response
Tracker}
(\href{https://github.com/OxCGRT/covid-policy-tracker}{źródło}). Zawiera
kilkanaście zmiennych określających stopień wykorzystania różnych
środków walki z epidemią poszczególnych państw, notowanych codziennie,
oraz obliczony na ich podstawie Wskaźnik Surowości (Stringency Index).

\hypertarget{zmienne}{%
\subsubsection{Zmienne}\label{zmienne}}

Pierwsza grupa zmiennych odnosi się do \textbf{zakazów i ograniczeń
nałożonych na obywateli}. Są to:

\begin{itemize}
\tightlist
\item
  {[}C1{]} zamykanie szkół i uniwersytetów: 0 - brak; 1 - rekomendowane;
  2 - nakazane zamknięcie określonego rodzaju (np. tylko szkół wyższych
  lub tylko publicznych); 3 - nakazane zamknięcie wszystkich rodzajów
  szkół;
\item
  {[}C2{]} zamykanie miejsc pracy: 0 - brak; 1 - rekomendowane; 2 -
  nakazane dla niektórych zawodów; 3 - nakazane dla wszystkich, z
  wyjątkiem niezbędnych zawodów;
\item
  {[}C3{]} odwołanie wydarzeń publicznych: 0 - brak; 1 - rekomendowane;
  2 - nakazane;
\item
  {[}C4{]} ograniczenia spotkań: 0 - brak; 1 - restrykcje powyżej 1000
  osób; 2 - restrykcje między 100 a 1000 osób; restrykcje między 10 -
  100 osób; restrykcje poniżej 10 osób;
\item
  {[}C5{]} zamykanie publicznego transportu: 0 - brak; 1 - rekomendowane
  (lub zmniejszanie częstotliwości kursóW); 2 - zamknięcie transportu
  (lub zakazanie większości obywateli korzystania z niego);
\item
  {[}C6{]} nakaz pozostania w domu: 0 - brak; 1 - zalecane; 2 -
  możliwość wyjścia tylko w koniecznych sytuacjach (zakupy, spacer z
  psem itd.); 3 - silne ograniczenia (np. możliwość wyjścia raz na kilka
  dni);
\item
  {[}C7{]} ograniczenia przemieszczania się (wewnątrz kraju): 0 - brak;
  1 - rekomendowane; 2 - ograniczone;
\item
  {[}C7{]} kontrole podróży międzynarodowych (z regionów wysokiego
  ryzyka): 0 - brak; 1 - testy; 2 - kwarantanna; 3 - zakaz podróży do i
  z obszarów wysokiego ryzyka; 4 - całkowite zamknięcie granic.
\end{itemize}

Zmienne C1 do C7 mają dodatkowo flagę oznaczającą zakres oddziaływania:
0 - regionalne lub stargetowane; 1 - obowiązujące ogólnie i w całym
kraju.

Druga grupa zmiennych dotyczy \textbf{postępowania państwa}:

\begin{itemize}
\tightlist
\item
  {[}H1{]} kampania informacyjna o COVID-19: 0 - brak; 1 -
  przedstawiciele władz ostrzegające o zagrożeniu i zachęcające do
  ostrożności; 2 - regularna kampania informacyjna;
\item
  {[}H2{]} polityka testowania: 0 - brak testów; 1 - testowanie
  wyłącznie ludzi z symptomami \textbf{oraz} spełniających dodatkowe
  kryteria (np. kontakt ze znanym przypadkiem, kluczowa grupa zawodowa,
  powracający zza granicy); 2 - testowanie każdego z symptomami; 3 -
  otwarte testy, dostępne dla każdego;
\item
  {[}H3{]} - śledzenie przypadków (z kim miały kontakt osoby zarażone):
  0 - brak; 1 - limitowane, dla niektórych przypadków; 2 - zawsze;
\item
  {[}E1{]} państwo wypłaca pensje obywatelom, którzy w wyniku restrykcji
  stracili pracę: 0 - brak; 1 - mniej niż 50\% utraconego lub
  przeciętnego wynagrodzenia; 2 - więcej niż 50\% utraconego lub
  przeciętnego wynagrodzenia;
\item
  {[}E2{]} zwolnienie gospodarstw domowych z długów lub innych
  zobowiązań (np. zamrożenie kredytów, zakaz odcinania niepłacącym wody
  itp.): 0 - brak; 1 - wąski zakres, jeden tym kontraktu; 2 - stosowane
  szeroko, dla wielu typów kontraktów.
\end{itemize}

Zmienna H1 ma również flagę oznaczającą zakres oddziaływania
regionalnego, natomiast E1 ma flagę zakresu formalnego: 0 - tylko dla
pracowników etatowych; 1 - dla wszystkich pracowników.

Wreszcie trzecia grupa dotyczy \textbf{działań w formie finansowej}:

\begin{itemize}
\tightlist
\item
  {[}E3{]} ekonomiczna/fiskalna stymulacja gospodarki, zarówno jako
  zastrzyk pieniędzy jak i ulgi podatkowe: wyrażona w USD;
\item
  {[}E4{]} pomoc finansowa udzielona innych krajom na walkę z epidemią
  COVID-19.
\item
  {[}H4{]} krótkoterminowa, dodatkowa inwestycja w służbę zdrowia:
  wyrażona w USD;
\item
  {[}H5{]} inwestycja w badania nad szczepionką: wyrażona w USD;
\end{itemize}

\hypertarget{wskaux17anik-surowoux15bci}{%
\subsubsection{Wskaźnik surowości}\label{wskaux17anik-surowoux15bci}}

Wskaźnik jest obliczany na podstawie pierwszej grupy oraz zmiennej H1.
Dla każdej zmiennej sumowana jest jej wartość oraz wartość flagi (jeśli
występuje) i dzielona przez maksymalną wartość, jaką da się dla niej
uzyskać. \textbf{Wsakźnik jest średnią ze wszystkich tak obliczonych
wartości.} Można go interpretować jako intensywność działań o
charakterze legislacyjnym, podejmowanych przez rządy.

Wykresy pokazują, jak wzrastał poziom wskaźnika surowości w różnych
krajach. Podzielono je na takie same grupy jak w poprzednich sekcjach.

\includegraphics{article_files/figure-latex/stringency-1.pdf}
\includegraphics{article_files/figure-latex/stringency-2.pdf}
\includegraphics{article_files/figure-latex/stringency-3.pdf}

W pierwszej grupie działania były intensyfikowane stopniowo: zaczynając
już od końca stycznia, największe wzrosty zaliczyły pod koniec marca i
ustabilizowały się w kwietniu. Przez większą część badanego okresu w
walce z epidemią przodowały Włochy, ostatecznie bardzo wysoką wartość
wskaźnika osiągnęły też Hiszpania i Francja. Belgia, Finlandia i Niemcy
prowadzą nieco łagodniejszą politykę kryzysową.

W drugiej i trzeciej grupie restrykcje są wprowadzane gwałtownie i
niemal w tym samym czasie (połowa marca). Ostatecznie wszędzie wskaźnik
osiąga podobne poziomy: 60 - 90\%

\hypertarget{styl-kryzysowy}{%
\subsubsection{Styl kryzysowy}\label{styl-kryzysowy}}

Oprócz intensywności działań, można również rozważać różnice w stylu
radzenia sobie z kryzysem. Zamiast jednego wskaźnika, obliczę dwa: dla
grupy restrykcji i zakazów (S1 - S4 i S6 - S7) oraz dla grupy działań
podejmowanych przez państwo (S5 i S12 - S13). Pozycja w przestrzeni
wyznaczonej przez te dwie osie pokazuje styl radzenia sobie z kryzysem
epidemii.

\includegraphics{article_files/figure-latex/strategy-1.pdf}

Można zauważyć, że państwa generalnie w większym stopniu sięgają po
restrykcje (dlatego osie nie zaczynają się w zerze). W lewym dolnym rogu
mamy państwa o względnie niskiej intensywności każdego rodzaju działań,
a w prawym górnym te, które stosują chętnie jeden i drugi. Zielona linia
wyznacza stosunek 1:1. Tylko dwa państwa (Austria i Luksemburg) chętniej
sięgają po strategię wspomagania niż restrykcji (Rumunia i Irlandia są
na granicy). Żółta linia ma nachylenie 1,5. Większość państw znajduje
się między nią a zieloną, co oznacza, że stosują obie strategie w
podobnym stopniu, ale jednak z preferencją dla restrykcji. Wreszcie
czerwona linia wyznacza stosunek 2:1. Polska, Grecja, Dania i Słowenia
to kraje, gdzie rząd ponad dwukrotnie intensywniej stosuje restrykcje
niż strategię wspomagania.

Jakie to ma znaczenie? Można argumentować, że w czasach zarazy
odpowiedzialni obywatele powinni ze zrozumieniem przyjąć nałożone na
nich ograniczenia, kierując się troską o zdrowie i życie innych ludzi.
Podążając za tą przesłanką, można pochwalać rządzących, którzy nie
wahają się sięgać po tak poważne ograniczenia swobody swoich suwerenów,
choć wiedzą, jaki to będzie miało skutek dla gospodarki i jak wielu
ludziom utrudni to życie.

Z drugiej strony, lekkość nakładania na obywateli restrykcji i zamykania
całych dziedzin ich życia może świadczyć o lekceważeniu przez władzę
sytuacji ``zwykłych ludzi'' czy patrzeniu na nich z góry. Zwłaszcza,
jeśli nie idzie w parze z równie intensywnymi działaniami ze strony
państwa.

\hypertarget{dziaux142ania-ekonomiczne}{%
\subsubsection{Działania ekonomiczne}\label{dziaux142ania-ekonomiczne}}

Osobno będziemy rozpatrywać zastrzyki finansowe.

Pierwsza zmienna dotyczy wydatków poniesionych na rzecz ratowania
gospodarki przed kryzysem, nieunikalnie związanym z epidemią.
\emph{Wydatki} mogą tu być zarówno zasiłkami jak i ulgami podatkowymi
(dlatego ich wartości potrafią być tak wysokie). Wyrażone są w
procentach PKB (z ostatniego kwartału 2019), dla czytelności ucięte na
poziomie 200 (etykiety przy słupkach przedstawiają realne wartości).

Ta forma reakcji na epidemię ma bardzo dużą wariancję. Od państw, które
nie korzystają z niej wcale, do wydających wielokrotność swojego PKB.

\includegraphics{article_files/figure-latex/fiscal-1.pdf}

Druga forma to wydaki bezpośrednio zasilające służbę zdrowia,
\emph{poza} normalnymi wydatkami na ten resort. Tym razem zostały
wyrażone w promilach PKB.

Widać, że wydatkami również rządzą rozmaite strategie. Są państwa, które
nie skorzystały z żadnej z tych form (jak Finlandia), a są takie, które
intensywnie wykorzystały obie (np. Cypr). Niektóre, (jak Polska i
Włochy), wydały stosunkowo dużo na gospodarkę i niewiele na służbę
zdrowia - inne skupiły się przede wszystkim na zdrowiu. Porównując te
liczby należy mieć w pamięci różnice skal. Na przykład Niemcy poświęcili
mniej więcej tyle samo na rzecz gospodarki i służy zdrowia.

\includegraphics{article_files/figure-latex/healthcare-1.pdf}

Inwestycje w szczepionkę mają nieco inny charakter, ponieważ nie są
działaniem bezpośrednio na rzecz własnego kraju, ale - wszystkich
cierpiących z powodu epidemii. Sześć państw, które zaangażowały się w te
badania, możemy umieścić w swoistej Hall of Fame zasłużonych dla
ludzkości. Są to: Hiszpania, Niemcy, Finlandia, Francja, Słowenia oraz
Szwajcaria.

\includegraphics{article_files/figure-latex/vaccines-1.pdf}

Podobnie jest w przypadku udzielania międzynarodowej pomocy:

\includegraphics{article_files/figure-latex/aid-1.pdf}

\hypertarget{tworzenie-wskaux17anikuxf3w}{%
\subsection{Tworzenie wskaźników}\label{tworzenie-wskaux17anikuxf3w}}

W poprzednich sekcjach starałam się przedstawić jak najszerzej sytuację
i uzasadnić dokonane wybory. Teraz spróbuję podsumować zebrane
informacje w następujących wskaźnikach:

\begin{itemize}
\tightlist
\item
  różnica między pozycją państwa pod względem czasu trwania epidemii i
  pod względem liczby przypadków - jako \textbf{wskaźnik rozwoju
  epidemii}. Jeśli jest dodatnia, państwo ma względnie silnie rozwiniętą
  epidemię; jeśli ujemna, udało się ją zahamować relatywnie dobrze;
\item
  stosunek śmiertelności do liczby przypadków na 1 mln mieszkanców -
  wskaźnik \textbf{jakości służby zdrowia}. Im gorzej służba zdrowia
  była przygotowana, tym jest wyższy;
\item
  stosunek stopnia ograniczeń nałożonych na obywateli do stopnia działań
  podjętych przez państwo - jako \textbf{wskaźnik stylu rządzenia w
  kryzysie}. Im wyższe wartości, tym bardziej ``autorytarny'' styl;
\item
  najgłębszy procentowy spadek głównego indeksu giełdowego w stosunku do
  końca roku 2019 - jako \textbf{wskaźnik kryzysu gospodarczego}. Im
  większa liczba (mniejszy spadek) tym lepiej;
\end{itemize}

\begin{longtable}[]{@{}lrrrr@{}}
\caption{Wartości wskaźników}\tabularnewline
\toprule
Państwo & Wskaźnik rozwoju epidemii & Wskaźnik jakości służby zdrowia &
Wskaźnik stylu zarządzania & Wskaźnik kryzysu
gospodarczego\tabularnewline
\midrule
\endfirsthead
\toprule
Państwo & Wskaźnik rozwoju epidemii & Wskaźnik jakości służby zdrowia &
Wskaźnik stylu zarządzania & Wskaźnik kryzysu
gospodarczego\tabularnewline
\midrule
\endhead
Spain & 4 & 0.0943301 & 1.3651316 & -0.3646672\tabularnewline
Belgium & 4 & 0.1033074 & 1.3020833 & -0.3625774\tabularnewline
Ireland & 13 & 0.0707490 & 1.0000000 & -0.3920545\tabularnewline
Luxembourg & 16 & 0.0310732 & 0.8894231 & -0.4595722\tabularnewline
Netherlands & 7 & 0.2664481 & 1.2951389 & -0.3316021\tabularnewline
Italy & -2 & 0.1723084 & 1.1820652 & -0.3663658\tabularnewline
France & -6 & 0.3193793 & 1.1684783 & -0.3723333\tabularnewline
Switzerland & 0 & 0.0521662 & 1.2500000 & -0.2313426\tabularnewline
Austria & -2 & 0.0600606 & 0.8593750 & -0.4849918\tabularnewline
Denmark & -1 & 0.1265321 & 2.1875000 & -0.4140856\tabularnewline
Romania & 2 & 0.3615343 & 1.0000000 & -0.2945035\tabularnewline
Estonia & -2 & 0.0514447 & 1.4062500 & -0.2409940\tabularnewline
Cyprus & 9 & 0.2114099 & 1.5104167 & -0.2878788\tabularnewline
Slovenia & 5 & 0.2973565 & 2.0192308 & -0.2590672\tabularnewline
Bulgaria & 6 & 1.3135909 & 1.0546875 & -0.2857394\tabularnewline
Finland & -13 & 0.0000000 & 1.3583333 & -0.3080268\tabularnewline
Greece & -6 & 0.7938070 & 2.5937500 & -0.4722104\tabularnewline
Portugal & -4 & 0.0423243 & 1.8125000 & -0.3132783\tabularnewline
Germany & -17 & 0.0046976 & 1.2664474 & -0.3628422\tabularnewline
Czech Republic & -5 & 0.1034510 & 1.4062500 & -0.3811837\tabularnewline
Poland & -4 & 0.6865862 & 3.1250000 & -0.3927091\tabularnewline
Hungary & -4 & 1.6667969 & 1.5357143 & -0.3606233\tabularnewline
\bottomrule
\end{longtable}

Wskaźniki można by zagregować średnią ważoną, żeby ustawić państwa w
kolejności od najlepiej radzącego sobie z epidemią do najgorszych.
Wybranie wag nie jest jednak łatwe, bo każdy wskaźnik ma inny charakter
i skalę. Wymagałoby to też decyzji, jak duże znaczenie dla ostatecznej
oceny mają poszczególne wskaźniki. Można by na przykład uznać, że
wskaźnik stylu zarządzania w kryzysie jest dużo mniej istotny niż
pozostałe, albo że gospodarka jest ważniejsza niż liczba przypadków,
skoro zdecydowana większość z nich nie jest poważnym zagrożeniem dla
zdrowia.

Zupełnie innym podejściem jest znalezienie \textbf{zbioru rozwiązań
optymalnych w sensie Pareto}. Każde państwo traktujemy jak jedno
rozwiązanie problemu epidemii, który chcemy optymalizować pod kątem
wymienionych wyżej czterech kryteriów. Rozwiązanie jest
\textbf{zdominowane}, jeśli można znaleźć inne rozwiązanie, które jest
lepsze pod względem jednego z kryteriów i lepsze bądź równe pod względem
wszystkich pozostałych.

Zbiór \textbf{niezdominowanych} rozwiązań jest optymalny w sensie Pareto
- nie ma jednego państwa, które radzi sobie absolutnie najlepiej, ale
każde z nich najlepiej realizuje swoją specyficzną strategię. Państwa,
które okazały się niezdominowane to: \textbf{Irlandia, Luksemburg,
Włochy, Francja, Szwajcaria, Austria, Rumunia, Estonia, Bułgaria,
Finlandia i Niemcy}.

Jeśli założymy, że wszystkie kryteria mają takie samo znaczenie, to
problem różnych jednostek i rzędów wielkości wskaźników można ominąć,
konstruując coś w rodzaju \emph{klasyfikacji medalowej}. Państwa są
rangowane od najlepszych do najgorszych w świetle każdego z kryteriów, a
następnie z tych rang obliczana jest średnia.

\begin{longtable}[]{@{}lrrrrrr@{}}
\caption{Klasyfikacja}\tabularnewline
\toprule
Państwo & Hamowanie epidemii & Jakość służby zdrowia & Styl zarządzania
& Gospodarka & Średnia & Pozycja\tabularnewline
\midrule
\endfirsthead
\toprule
Państwo & Hamowanie epidemii & Jakość służby zdrowia & Styl zarządzania
& Gospodarka & Średnia & Pozycja\tabularnewline
\midrule
\endhead
Finland & 2 & 1 & 12 & 7 & 5.50 & 1\tabularnewline
Germany & 1 & 2 & 9 & 12 & 6.00 & 2\tabularnewline
Switzerland & 13 & 6 & 8 & 1 & 7.00 & 3\tabularnewline
Estonia & 9 & 5 & 14 & 2 & 7.50 & 4\tabularnewline
Portugal & 6 & 4 & 18 & 8 & 9.00 & 5\tabularnewline
Austria & 9 & 7 & 1 & 22 & 9.75 & 6\tabularnewline
France & 3 & 17 & 6 & 15 & 10.25 & 7\tabularnewline
Romania & 14 & 18 & 3 & 6 & 10.25 & 7\tabularnewline
Italy & 9 & 13 & 7 & 14 & 10.75 & 9\tabularnewline
Czech Republic & 5 & 11 & 14 & 16 & 11.50 & 10\tabularnewline
Belgium & 15 & 10 & 11 & 11 & 11.75 & 11\tabularnewline
Luxembourg & 22 & 3 & 2 & 20 & 11.75 & 11\tabularnewline
Bulgaria & 18 & 21 & 5 & 4 & 12.00 & 13\tabularnewline
Ireland & 21 & 8 & 3 & 17 & 12.25 & 14\tabularnewline
Spain & 15 & 9 & 13 & 13 & 12.50 & 15\tabularnewline
Netherlands & 19 & 15 & 10 & 9 & 13.25 & 16\tabularnewline
Cyprus & 20 & 14 & 16 & 5 & 13.75 & 17\tabularnewline
Slovenia & 17 & 16 & 19 & 3 & 13.75 & 17\tabularnewline
Hungary & 6 & 22 & 17 & 10 & 13.75 & 17\tabularnewline
Denmark & 12 & 12 & 20 & 19 & 15.75 & 20\tabularnewline
Greece & 3 & 20 & 21 & 21 & 16.25 & 21\tabularnewline
Poland & 6 & 19 & 22 & 18 & 16.25 & 21\tabularnewline
\bottomrule
\end{longtable}

Na podium znalazły się Finlandia i Niemcy (oba wypadły najlepiej w
kategoriach ``medycznych'', a przeciętnie w pozostałych) oraz Szwajcaria
(z najmniejszym spadkiem na giełdzie i niezłym miejscem pod względem
jakości służby zdrowia). Polska sprawdziła się dobrze w hamowaniu
epidemii, ale wypadła bardzo słabo we wszystkich pozostałych
kategoriach, więc zajęła niechlubne ostatnie miejsce, razem z Grecją.

W wynikach nie widać wyraźnych wzorców związanych z położeniem
geograficznym, zamożnością czy ludnością państw. Ciekawe jest jednak to,
że cztery z pięciu państw udzielających pomocy międzynarodowej i pięć z
sześciu państw inwestujących w tworzenie szczepionki znalazło się w
pierwszej dziesiątce - może hojność jednak popłaca?

\hypertarget{podsumowanie}{%
\subsection{Podsumowanie}\label{podsumowanie}}

To już wszystko. Rozgryzanie tych danych było ciekawym doświadczeniem i
przyznam, że gdybym miała przed tym zgadywać wyniki klasyfikacji,
wypadłabym bardzo słabo.

Mam przede wszystkim nadzieję, że ta analiza pokaże wam, że przejście od
prostych liczb do wniosków o ich przyczynach nie jest ani łatwe, ani
oczywiste. Po drodze trzba postawić szereg założeń, dotyczących zarówno
rzeczywistego świata jak i własnych priorytetów i przekonania o tym, co
jest korzystne, a co nie.

\end{document}
